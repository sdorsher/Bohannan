To count the number of processor steps required, the algorithms
described in Section~\ref{subsec:avgShift} are converted to pseudocode
then the conversions listed in Table~\ref{tab:stepCounting} are
applied.

\begin{table}
\begin{tabular}{ll}
\hline
Type    &Cost   \\
\hline
Integer assign  &Free\\
Floating point assign &Free \\
Negation &Free \\
Integer $+$,$-$ &Free \\
Floating point $+$,$-$ &1 \\
Integer $*$,$\/$ &1 \\
Floating point $*$,$\/$ &1 \\
Floating point exponentiation &1 \\
If (int clause) &Free \\
If (float clause) &1 \\
While (int clause) &1 \\
While (float clause) &2 \\
\hline
\end{tabular}
\label{tab:stepsCounting}
\caption{Number of processor steps required for various operations.}
\end{table}

During intialization, several things must happen. The array defining
the binning structure, $b_k$, must be set. The history of the input
signal must be zeroed and the first output must be set to zero as
well. Most significantly, the binned weights must be calculated for
the chosen binning structure.

The time it takes to calculate or set $b_k$ is moderately dependent
upon the details of the binning structure chosen. However, any binning
structure will have a processing time that scales as $N_b$ rather than
$N_t$ because of the number of bins that must be set. Here $N_t$ is
the maximum history depth in time-steps of the $N_b$ memory storage
bins. By definition,

\begin{equation}
N_t = \displaystyle\sum_{k=0}^{N_b}b_k.
\label{eqn:Nt}
\end{equation}

Since zeroing (assignment) is computationally free, the bulk of the
initialization process occurs during the computation of the
weights. Since each individual Grunwald weight must be summed, this
process scales as $N_t$. Equation~\ref{eqn:Wk} gives the algorithm for
the calculation of the bin weights. The following recursion formula
for Grunwald weights can be used to iterate $w_j$ to a new value with
each time step. In the $j$th time step, $w_j$ has the value

\begin{equation}
w_j = \frac{j-1-\alpha}{j}w_{j-1}.
\label{eqn:GrunwaldRecursion}
\end{equation}

In the implementation of an algorithm for initialization of the
weights, individual Grunwald weights $w_j$ need not be stored in an
array after calculation. These individual weights can be summed
immediately into the corresponding bin weights $W_k$ according to
Equation~\ref{eqn:sumWk}, at which time the value of $w_j$ may be safely
overwritten with $w_{j+1}$.

Using these formulas, pseudocode for the weight-initialization
algorithm is given in Algorithm~\ref{alg:initializing}.

\begin{algorithm}
\caption{Initialization of the binned weights. $\alpha$ is the order of the differintegral and $b_k$ is the bin capacity of the $k$th bin.}
\label{alg:initializing}
\begin{algorithmic}
%\REQUIRE $0 \le \alpha \le 1$
%\ENSURE Equation~\ref{eq:Wk}
\FOR{$0\le k < N_b$}
\STATE $W_k \gets 0.0$
\ENDFOR
\STATE $j \gets 1$
\STATE $k \gets 0$
\STATE $w \gets 1.0$
\STATE $p \gets b_0$
\WHILE{$k < Nb$}
\STATE $W_k \gets W_k + w$
\STATE $j \gets j +1$
\IF{$j>p$}
\STATE $k \gets k +1$
\STATE $p \gets p +b_k$
\ENDIF
\STATE $w \gets \frac{j-1.0-\alpha}{j}w$
\ENDWHILE
\end{algorithmic}
\end{algorithm}
Using Table~\ref{tab:stepsCounting}, the number of processor steps
required to implement the weight initialization algorithm is $5N_t+1$.

\smallskip

Equation~\ref{eqn:updating} may be used as the basis for an algorithm
to update the binned average input signal history values when a new
input signal value is read. Care must be taken to account for the
filling of the bins until the approximation reaches steady-state.

Again, a memory saving opportunity is available. Since only one bin
can be filling at a time and the rest must be either full ($c_k=b_k$)
or empty ($c_k=0$), the full array of $N_b$ occupancy numbers $c_k$
need not be stored. Instead, it is sufficient to store the index of
the bin that is filling, $k_{filling}$ and its occupancy,
$c_{filling}$. These unfilled bins with $k>k_{filling}$ and $c_k=0$
are not updated when a new input signal value is read into the
history.

Pseudocode for the bin-average updating algorithm is given in
Algorithm~\ref{alg:updating}.

\begin{algorithm}
\caption{Algorithm for updating the bin average values of the input signal history. $X_{k}$ is the $k$th average binned value of the history. $x_0$ is the input signal value that has just been read into the digital fractor.}
\label{alg:updating}
\begin{algorithmic}
%\STATE $k_{filling}\gets 0$
%\STATE $c_{filling}\gets 0$
\STATE $k \gets N_b$
\WHILE{$k>0$}
\IF{$k=k_{filling}$}
\STATE $c_{filling}\gets c_{filling} +1$
\IF{$c_{filling}>b_k$}
\STATE $k_{filling} \gets k_{filling} +1$
\ENDIF
\STATE $k\gets k+1$
\STATE $X_k \gets \left(1.0-\frac{1.0}{c_{filling}}\right)X_k 
- \frac{1.0}{c_{filling}}X_{k-1}$
\ELSIF{$k<k_{filling}$}
\STATE $X_k \gets \left(1.0-\frac{1.0}{b_k}\right)X_k 
- \frac{1.0}{b_k}X_{k-1}$
\ENDIF
\STATE $k\gets k -1$
\ENDWHILE
\IF{$k_{filling}=0$}
\STATE $c_{filling} \gets c_{filling} +1$
\IF{$c_{filling}>b_0$}
\STATE $k_{filling} \gets 1$
\STATE $c_{filling} \gets 1$
\ENDIF
\STATE $X_k \gets \left(1.0-\frac{1.0}{c_{filling}}\right)X_0 
- \frac{1.0}{c_{filling}}x_{0}$
\ELSE
\STATE $X_k \gets \left(1.0-\frac{1.0}{b_0}\right)X_0 
- \frac{1.0}{b_0}x_{0}$
\ENDIF
\end{algorithmic}
\end{algorithm}
Table~\ref{tab:stepsCounting} yields $12N_b$ processor steps
required to update the bin average values of the input signal history
when a new data point is read.

\smallskip

Our algorithm for differintegration is based upon
Equations~\ref{eqn:avgSimpleGrunwald}~and~\ref{eqn:Wbar}. The
algorithm is shown in Algorithm~\ref{alg:modGrun}.

\begin{algorithm}
\caption{The algorithm for differintegration using the modified Grunwald method described in Section~\ref{sec:avgShift}}.
\label{alg:modGrun}
\begin{algorithmic}
\STATE $k\gets N_b$
\WHILE{$k\ge 0$}
\IF{$k=k_{filling}$}
\STATE $W_k\gets\frac{c_{filling}}{b_k}W_k$
\ENDIF
\STATE $D_t^\alpha \gets D_t^\alpha + W_kX_k$
\ENDWHILE
\STATE $D_t^\alpha \gets (\Delta t)^{-\alpha}D_t^\alpha$
\end{algorithmic}
\end{algorithm}
Based upon Table~\ref{tab:stepsCounting}, the number of processor
steps required to calculate the differintegral using the modified
Grunwald approach presented in this paper is $3N_b+2$.

\begin{table}
\begin{tabular}{ll}
\hline
Stage &Processor steps required \\
\hline
Initialization of weights &$5N_t+1$ \\
Updating the average bin history values &$12N_b$ \\
Summing the differintegral &$3N_b+2$\\
\hline
\end{tabular}
\caption{Processor steps required for the different stages of the average Grunwald algorithm.}
\label{tab:processor}
\end{table}

\smallskip

The number of processor steps required for the three stages are
summarized in Table~\ref{tab:processor}. Memory requirements for the
algorithms described in this section are shown in
Table~\ref{tab:memory}. 

The two real-time operation stages, updating and differintegrating,
together require $15N_b+2$ processor steps. With 26 bins ($N_b=26$),
$392$ processor steps are required for each time-step of a fractional
order differentiation or integration. $3N_b+6=84$ memory registers are
required for 26 bins of history.

Initialization, on the other hand, is a very resource-consuming
process. For an exponentially scaled binning structure with a scale
factor of two, $N_t=2^{26}-1=67,108,863$ for 26 bins. Initialization
therefore requires $3.36\times 10^8$ processor steps, a calculation
perhaps best achieved by a computer in a one-time operation. 


\begin{table}[h]
\begin{tabular}{llll}
\hline
Variable& Definition & Size& Stages \\
\hline
$b_k$ & Bin capacity & $N_b$ &I, U, D \\
$W_k$ & Binned weights & $N_b$ &I, U, D \\
$X_k$ & Binned history & $N_b$ &I, U, D \\
$x_0$ & Current input data & 1 & U \\
$D_t^\alpha$ & Output differintegral & 1 & I, D\\
$\alpha$ &Order of differintegral &1 &I, D\\
$N_b$ &Number of bins &1 &I,U,D\\
$N_t$ &Number of time steps stored &1 & I\\
$\Delta t$ &Time step &1 &D\\
$w_j$ & Grunwald weight, temp variable & 1 & I\\
$k_{filling}$ & Currently filling bin & 1 & U, D\\
$c_{filling}$ & Occupancy of filling bin & 1 & U, D \\  
\hline
\hline
& Total w/o initialization &$3N_b+6$ &\\
 & Total & $3N_b+9$ &\\
\hline
\end{tabular}
\label{tab:memory}
\caption{Accounting for memory in the Average-Shift algorithm. I = Initialization, U=Updating, D = Differintegration}
\end{table}


