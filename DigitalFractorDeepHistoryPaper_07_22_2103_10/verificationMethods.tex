While the algorithms described in Section~\ref{sec:algorithmDefn} are
ultimately intended for use in a digital fractor, in order to
characterize their frequency response we have simulated them in C++ on a
personal computer. The relative phase as a function of frequency is based upon the
fractional derivative or integral of sinusoidal inputs, sweeping from
low frequency to high frequency in logarithmic steps. Relative phase
is reconstructed by curve fitting (adapted from Numerical Recipes in C~\cite{Press:92}), using the last
half of the output signal to the sine and cosine components of a
sinusoidal signal with the same frequency as the input signal. From that, phase and amplitude are calculated.

To verify phase accuracy, we examine the phase shift of a single
crossing in the time domain for an input frequency of $0.7924$ Hz
(Figures~\ref{fig:timeDomain}). Table~\ref{tab:phaseRecon}
shows a comparison of reconstructed phases using the sinusoidal fit
method and phases obtained from Figure~\ref{fig:timeDomain} by
measuring the phase difference in a single zero-crossing. The results demonstrate that the reconstructed phases are accurate when the output phase is within about ten degrees of the expected phase. Since we are interested in the deviations of the phase from the expected phase by about ten degrees, this should be sufficient. Transients may account for the remainder of the discrepancy in phase in the cases where the single crossing phase and the fit phase differ more dramatically. 


\begin{figure}
\epsfig{file=FIG3.eps, height=130mm, width = 90mm}
\caption{Zero crossings of a $0.5$ order derivative of a sine wave input signal
at long times relative to the period. From this the reconstructed
  phase can be verified using the knowledge that the frequency of this
  sine wave was $0.7924$ Hz. Reconstructed and zero-crossing phases
  are shown in Table~\ref{tab:phaseRecon}. In this plot, the duration
  was $10.0$ s and there were $1000$ time steps such that $\delta t=0.01$ s. There were 10 terms
  in the IIR and 10 bins in all binned Gr{\"u}nwald approximations. A
  variety of binning methods were used: truncated simple Gr{\"u}nwald
  (GL), linearly increasing bin sizes (LIN), bin sizes increasing as a
  square (SQ), exponentially increasing bin sizes (EXP), and bin sizes
  following a Fibonacci rule (FIB).}
\label{fig:timeDomain}
\end{figure}

\begin{table}
\begin{tabular}{llllllll}
\hline
Name &Reconstructed (degrees) &Zero-crossing (degrees)\\ IIR40 &$45.0$
&$45\pm 2$\\ GL &$44.2$ &$41\pm 2$\\ GL26 &$23.0$ &$32\pm 2$\\ LIN26
&$46.1$ &$41\pm 2$\\ SQ26 &$44.4$ &$41\pm 2$\\ EXP26 &$42.1$ &$41\pm
2$\\ FIB26 &$43.0$ &$41\pm 2$\\
\hline
\end{tabular}
\caption{Phases reconstructed through sinusoidal fits and phases approximated through single zero-crossings of a single frequency ($0.7924$ Hz) based upon Figure~\ref{fig:timeDomain}, with an expected phase of $45^\circ$.}
\label{tab:phaseRecon}
\end{table}

The simulation has successfully been run for up to $1,000,000$ time
steps, limited primarily by memory considerations due to the storage
necessary for phase and amplitude reconstruction. At this long
simulation duration, amplitudes and phases over six decades in
frequency are obtained.
