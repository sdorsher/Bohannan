
Interest in the application of the fractional calculus has been
growing at an ever increasing rate. Of long term and continued
interest is in the use of fractional order (FO) operators, such as
integrators and differentiators, in motion control
applications.~\cite{Luo:13} While wide bandwidth analog controllers
have been successfully demonstrated~\cite{Bohannan:08}, fractional
order analog circuit elements are not generally available and the
prototypes that have been demonstrated do not have the ability to be
retuned for a specific desired phase.~\cite{Monje:10}

Unfortunately, the existing techniques for digital approximation of FO
operators are limited in bandwidth to on the order of three and a half
decades of frequency response while nonlinear effects in motion
control systems can span five decades or more. This places a severe
constraint on the design and implementation of digital FO controllers,
i.e. how to set the sampling frequency to meet the high speed
requirements necessitated by the Nyquist sampling rate while at the
same time providing enough deep memory to get low offset
error. Additionally, the infinite impulse response type of
implementation cannot guarantee stability due to the limitations of
finite precision arithmetic. See e.g.~\cite{Chen:04a}.

Given the current necessity to implement FO controls in digital form,
it is desirable to obtain the most efficient algorithm to compute a
fractional order operator while maximizing the numerical stability of
the algorithm. Efficiency to be measured in both memory utilization
and number of computations per time step. This paper outlines a
computational method inspired by the Riemann-Liouville integral
definition and the Gr{\"u}nwald summation formula.~\cite{OldSpan:74} 
The essential concept is the rescaling of time by successive accumulation
of older data into increasing size bins for deeper memory.

Our motivation is two--fold. First to introduce the idea of ``time scaling'' 
for computations requiring coverage of very long time spans, and 
second to offer an algorithm that might be used in a microprocessor
based control unit (mcu) requiring flat phase response over a 
wide bandwidth. One might wonder why the attempt at such extended
bandwidth since integer order derivative and integral operator will typically
either decimate or expand the amplitude beyond the range of analog
to digital converters due to the $\omega^{\pm 1}$ scaling. 
Fractional order scaling is much ''softer'' in the sense that the magnitude
scaling only goes as $\omega^{\pm \alpha}$ with $|\alpha | < 1$, 
allowing for effective operation over a much wider bandwidth.
