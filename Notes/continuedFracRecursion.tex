\documentclass{article}
\usepackage{amssymb, amsmath}
\begin{document}

\Section{Motivation} 

In {\em Two direct Tustin discretization methods for fractional-order
  differentiator/integrator}, Vinagre at al write that the discrete
transfer function approximating fractional order control systems is
given by their equation 11:

\begin{eqnarray}
D_{\pm r}(z)&= \frac{Y(z)}{F(z)} =\left(\frac{2}{T}\right)^{\pm r}
CFE\left{\left(\frac{1-z^{-1}}{1+z^{-1}}\right)\right}_{p,q}\\
&=\left(\frac{2}{T}\right)^{\pm r}\frac{P_p(z^{-1})}{Q_q(z^{-1})}
\end{equation}

\noindent The purpose of this note is to find the continued fraction
expansion (CFE) and the corresponding polynomial ratio expression to
the order $p=q$.

\section{Continued fraction expansion}

\noindent For our purposes, we can use the following expansion from
page 107 of {\em Expansions of Certain Functions}, equation 2.13:

\begin{equation}
\left(\frac{x+1}{x-1}\right)^\nu = 1 + \frac{2\nu}{x-\nu+}
\frac{\nu^2-1}{3x+}\frac{\nu^3-4}{5x+}\cdots\frac{\nu^2-n^2}{(2n+1)x+}\cdots
\end{equation}

\noindent Dividing both the top and the bottom by x and replacing
$\nu\rightarrow\mp r$, we obtain

\begin{equation}
\left(\frac{1-z^{-1}}{1+z^{-1}}\right)^{\pm r} = 1 \mp \frac{2r}{x\pm r+}
\frac{r^2-1}{3x+}\frac{r^2-4}{5x+}\cdots\frac{r^2-n^2}{(2n+1)x+}\cdots
\label{eqn:CFE}
\end{equation}


\section{Continued fraction recursion relations}

\noindent According to the {/em CRC Concise Encyclopedia of
  Mathematics} by Eric Weisstein, a continued fraction of the form

\begin{equation}
x = b_0 + \frac{a_1}{b_1 + \frac{a_1} + {b_2 + \frac{a_3} {b_3 + \cdots}}}
\label{eqn:continuedFrac}
\end{equation}

\noindent can be expressed as a polynomial quotient $\frac{P_n}{Q_n}$
when it is truncated at the $n$th index. $P_n$ and $Q_n$ can be
obtained from the following recursion relations:

\begin{eqnarray}
P_{-1}=1 \\
Q_{-1}=0\\
P_0=b_0\\
Q_0=1\\
P_j=b_jP_{j-1}+a_jP_{j-2}\\
Q_j=b_jQ_{j-1}+a_jQ_{j-2}
\end{eqnarray}

\section{Using the recursion relations}

We can 
